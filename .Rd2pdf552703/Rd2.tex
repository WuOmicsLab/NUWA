\nonstopmode{}
\documentclass[letterpaper]{book}
\usepackage[times,inconsolata,hyper]{Rd}
\usepackage{makeidx}
\usepackage[utf8]{inputenc} % @SET ENCODING@
% \usepackage{graphicx} % @USE GRAPHICX@
\makeindex{}
\begin{document}
\chapter*{}
\begin{center}
{\textbf{\huge Package `NUWA'}}
\par\bigskip{\large \today}
\end{center}
\inputencoding{utf8}
\ifthenelse{\boolean{Rd@use@hyper}}{\hypersetup{pdftitle = {NUWA: A pipeline for robust inference of missing cell markers and deciphering immune cell fractions using mass spectrometry proteomics}}}{}
\begin{description}
\raggedright{}
\item[Type]\AsIs{Package}
\item[Title]\AsIs{A pipeline for robust inference of missing cell markers and deciphering immune cell fractions using mass spectrometry proteomics}
\item[Version]\AsIs{1.0}
\item[Date]\AsIs{2021-2-4}
\item[Author]\AsIs{Yuhao Xie }\email{xieyuhao@pku.edu.cn}\AsIs{}
\item[Maintainer]\AsIs{Lihua Cao }\email{lihuacao@bjcancer.org}\AsIs{}
\item[Description]\AsIs{NUWA is a computational pipeline for robust abundance inference of missing cell type markers in proteomic profiles (by NUWA-ms) and deciphering makeup of immune cell subsets (by NUWA-eDeconv or other popular deconvolution algorithms), which could enable accurate proteomic deconvolution of tissue-infiltrating cell populations. The performance of NUWA pipeline has been systematically evaluated by multiple approaches with validation using scRNA-seq data, see the NUWA manuscript for more details.}
\item[License]\AsIs{MIT}
\item[Encoding]\AsIs{UTF-8}
\item[LazyData]\AsIs{TRUE}
\item[RoxygenNote]\AsIs{7.3.1}
\item[URL]\AsIs{}\url{https://github.com/WuOmicsLab/NUWA}\AsIs{}
\item[Depends]\AsIs{R (>= 3.6.1),}
\item[biocViews]\AsIs{}
\item[Imports]\AsIs{e1071 (>= 1.7.3),
preprocessCore,
parallel,
abind,
reshape2,
ggplot2,
glmnet,
EPIC,
MCPcounter}
\item[Suggests]\AsIs{knitr,
roxygen2,
testthat,
rmarkdown}
\item[VignetteBuilder]\AsIs{knitr}
\item[Remotes]\AsIs{GfellerLab/EPIC,
ebecht/MCPcounter/Source,
dviraran/xCell}
\end{description}
\Rdcontents{\R{} topics documented:}
